\documentclass[a4paper,10pt]{article}
%\documentclass[a4paper,10pt]{scrartcl}

\usepackage[utf8]{inputenc}
\usepackage{german}

\title{Benutzer Handbuch}
\author{}
\date{}

\pdfinfo{%
  /Title    ()
  /Author   ()
  /Creator  ()
  /Producer ()
  /Subject  ()
  /Keywords ()
}

\begin{document}
\maketitle
\tableofcontents
\newpage

\section{Redaktionsansicht}
In diesem Kapitel wird vorausgesetzt, daß die Redaktionsansicht eingestellt ist. Um die Redaktionsansicht einzustellen, kann man auf den Reiter Konto und dann auf den Button ``Redaktionsansicht'' clicken.

\subsection{Register}
Reiter Register\slash Thesaurus.

\subsubsection{Einträge zusammenführen}
Sobald mindestens zwei der Checkboxen vor den Registereinträgen gecheckt sind erscheint unterhalb der Überschrift der Link ``Registereinträge zusammenführen''. 
Wird dieser Link geclickt, so werden die gecheckten Registereinträge zusammengeführt. \textbf{Der (zeitl.) zuerst gecheckte Registereintrag überlebt}.\\
Alle Referenzen (Schlagworte, Metadatenfelder) die auf die gestorbenen Registereinträge verwiesen hatten, verweisen nun auf den zuerst gecheckten, überlebenden Registereintrag.

\subsubsection{Einträge verschieben}
Click auf das Symbol mit den drei Punkten (``mehr erfahren''). Dann Click auf ``Weiteren bestehenden Elternknoten für diesen Eintrag festlegen'' und in das Formular die ID des Registereintrags schreiben unter welchen der Eintrag verschoben werden soll.\\
Anschliessend kann ``
Zuordung zu diesem Elternknoten entfernen (löscht nicht den Eintrag)'' geclickt werden, wodurch der zu verschiebende Eintrag von seinem ursprünglichen Elternknoten gelöst wird.\\
Durch clicken auf ``
Zuordung zu diesem Elternknoten entfernen (löscht nicht den Eintrag)'' wird nur die Beziehung zum Elternknoten nicht der Eintrag gelöscht!


\end{document}
